\documentclass{article}

\usepackage{amsthm}
\usepackage{amsmath}
\usepackage{cite}
\usepackage{listings}
\usepackage{multicol}
\usepackage{booktabs}
\usepackage{url}

\setlength{\parindent}{4em}
\setlength{\parskip}{1em}

\begin{document}

\pagenumbering{gobble}
%% \maketitle
%% \newpage
%% \pagenumbering{arabic}

\begin{center}
  \textbf{Game AI Project Report}

  \textit{Derick Anderson}
\end{center}

\subsubsection*{The code}

All of the code can be accessed on GitHub \footnote{https://github.com/dandersonw/gai-final-project}.

\section*{Introduction}

Pentago \footnote{https://en.wikipedia.org/wiki/Pentago} is a two player board
game that we developed a deep reinforcement learning agent for. The game is
attractive because it is relatively tractable, but of enough complexity to prove
interesting.

The game is played similar to Connect 4, but on a 6x6 board where tokens can be
placed in any blank space. On a player’s turn, they must both place a token in a
blank space, then rotate one of the corner sub-boards 90 degrees in either
direction. Testing for a win state only occurs after the board has been rotated
and the turn has been concluded.

Originally we intended to do TD learning, but we later decided to implement
the AlphaZero methodology instead.

\subsubsection*{Topics}

Originally the point breakdown was 2 for neural network,
and 3 for the TD learning.
I believe that the AlphaZero method should count for the same as TD learning.

\section*{Team}

The team members were Derick Anderson and Steve Krueger.
We did not initially have a plan for distributing work,
and Steve pled that he was otherwise occupied,
so I (Derick) ended up implementing an entire first draft of the project
before Steve got started.
I asked him to work on optimizing the training process
(performance, hyperparameters),
but he then continued to not spend a lot of time on the project,
and what code he produced was unusable due to bugs.
He has some claim that I implemented a lot of the low hanging fruit,
leaving less easy stuff for him to do,
but the timeline of the project meant that I felt I had to implement
or the project would never be done.
Furthermore,
the buggy nature of the code he did produce did not make me confident
that waiting for him to implement would bear fruit.

\section*{Instructions}

\begin{itemize}
\item You need to create an isolated Python environment.  I recommend Anaconda
  \footnote{https://www.anaconda.com/}.
\item With that environment activated, navigate to the
  \texttt{gai-final-project/pentago} directory, and run \texttt{pip install -e .}
\item Run one of the commands in the README
  \footnote{https://github.com/dandersonw/gai-final-project/tree/master/pentago}
  to play against an agent or tally the results of agents playing.
\end{itemize}

\section*{Systems}

First I implemented minimax with a hand-written evaluation function.
No real design was necessary,
but careful implementation using compilable-to-machine-code Python was necessary
to get playable speed.

Then I implemented my janky AlphaZero clone.
That included a few components:
a residual convolutional neural network,
a Monte-Carlo tree search implementation,
and a training procedure.
The main series of design decisions I had to make
was how to pare down the process so that I could train it with my resources.

The most obvious decisions were to use a baby size network
(5 layers and 64 filters),
and to reduce the rollouts in the training from 800 to 400.
I took a page out of AlphaGo's book and implemented data augmentation
based on the rotation-invariance of board states,
I implemented on-machine distributed self-play and evaluation
in order to help speed up training.

More difficult decisions had to do with the training process.
I had to decide how many self-play games to create per round,
how much to train the network per round,
how to set the temperature during games,
and many more questions.
I read one series of blog posts about implementing AlphaZero on
Connect 4 \footnote{https://medium.com/oracledevs/lessons-from-implementing-alphazero-7e36e9054191},
and otherwise tried to gleam insight from the original paper,
but mostly I tried some values until the process seemed to work.

One way in which I deviated quite severely from the AlphaZero training
is that I did use minimax with my hand-coded evaluation function
to start the network off learning.
That is,
I put a thousand games worth of minimax play into the memory
before I started the network learning.
I think that was necessary because
I was doing so few (30) games of self-play per learning round
that the network could overfit hard in the first epochs.

\section*{Failures and Setbacks}

My main problem was that I didn't have the compute resources I needed
to try many variations of the algorithm,
and to train a serious model once I found hyperparameters that seemed to work.

One particular failure was my failure to
make self-play game generation more efficient by splitting out
many worker threads to run MCTS and an inference service to evaluate boards.
The distributed processing framework I chose and
got my first-draft distributed self-play working in wasn't designed for such a task,
and I couldn't put in another framework/serving application
without engineering time I didn't have.

\section*{Successes}

The system does, basically, learn how to play the game.
When I run the training process
the agent does learn how to beat itself,
update the best agent checkpoint,
and learn how to beat itself again.
It furthermore comfortably beats the minimax baseline
with order-of-magnitude-comparable thinking time.

Speaking of reproducing that success,
the training process is conducted in the Jupyter notebook
in the \texttt{notebooks/} directory.

\section*{Evaluation}

The comments on our proposal said that we didn't have to do formal evaluation,
but I thought it prudent to do some informal evaluation.
The neural agent (with more than a few rollouts) always beats a random-move agent.
I've tabulated some stats for
the neural agent vs. minimax + hand-coded evaluation function.
You can reproduce these by following the instructions in the README.

\begin{table}[h]
  \centering
  \begin{tabular}[]{r | c c c c}
    & \multicolumn{4}{c}{\text{MCTS rollouts}}\\
    \cmidrule{2-5}
    Minimax Depth & 1 & 100 & 400 & 800\\
    \midrule
    1 & 7/93/0 & 34/65/0 & 66/34/0 & 70/30/0 \\
    2 & 2/98/0 & 50/50/0 & 63/36/1 & 66/33/1 \\
  \end{tabular}
  \caption{Win/Lose/Draw rates for neural agent varying minimax depth and MCTS rollout count}
\end{table}

The above reported results are from only one run each,
so some noise is to be expected.
Perplexingly,
while the trend of increasing rollouts uniformly appears
to increase the neural agent's win rate,
increasing the minimax depth doesn't have a uniform effect.

It is my subjective impression that the rollout numbers above are playable
as an opponent for a human,
and that the thinking time for depth 2 minimax and 400 rollouts MCTS
are within an order of magnitude of each other.
The worst-case branching factor is 288,
so depth 3 minimax is way too slow to play against.

\section*{Lessons Learned}

I learned something about project and team management from this experience.
Mostly I've been blessed with helpful partners up to this point,
so I had assumed that collaboration would just happen,
but in this case I shouldn't have assumed that.

I already knew about it,
but I was again reminded of the crushing difference
between the tens of thousands of TPUs that Google can mobilize
and the one CPU that I can.

\end{document}

%%% Local Variables:
%%% mode: latex
%%% TeX-master: t
%%% End:
